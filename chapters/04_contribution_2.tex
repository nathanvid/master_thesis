\section{Internet of Plants}

\subsection{Overview}

The Internet of Plants, also called IoP, is a concept that aims to interconnect the plant device previously built.
This in order to empower the device capabilities and to provide a better user experience.
This project includes:
\begin{itemize}
    \item A better sound quality by using professional sonification software
    \item The ability to create a full artistic experience by creating a distributed instrument
    \item Refining the interaction with the plant by using more complex data analyses
\end{itemize}



\subsection{Communication}

The communication system is based on WiFi technology. The ESP32 has specific wireless capabilities.
The server and the ESP32 devices are connected to the same network.
The ESP32 sends the raw data to the server using IP\footnote{Internet Protocol}. This is done
through a TCP \footnote{Transmission Control Protocol} socket open between the server and one ESP32.
The server can open as many socket as there are clients. The data sent using a string.
A "#" start character and a ";\\n" stop character are used to prevent the messages to 
be truncated and still processed on the server side. The IP protocol (whether it is on WiFi
or Ethernet) has been chosen for this application for several reasons:

\begin{itemize}
    \item Allow high bandwidth
    \item Available in most places
    \item Allow connection of multiple devices
    \item Already available on the server and on the ESP32
\end{itemize}

Other communication protocol have been benchmarked. Here is a table that summarizes
the choice: 

\begin{table}[h!]
    \begin{tabular}{lllll}
    Protocol                            & IP (WiFi/Ethernet) & Bluetooth & BLE   & Zigbee \\
    Handle multiple connections         & Yes                & No        & No    & Yes    \\
    Requires additional hardware        & No                 & No        & No    & Yes    \\
    Subject to interference             & Yes                & Few       & Few   & Yes    \\
    Energy efficiency (using a battery) & Days               & Months    & Years & Years 
    \end{tabular}
\end{table}

The result of this table confirms that WiFi is the right choice for our specific application.
\subsection{Server}

The server is a small fanless computer running Lubuntu. 

\subsection{Deployment and application}

\subsubsection{Distributed instruments}

\subsubsection{Art exhibition}

\subsection{Conclusion}

