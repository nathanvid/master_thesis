\section{Conclusion}

In wrapping up our study on the Internet of Plant project, we've uncovered insights into how people might interact with plants in a future where they make music through touch. Participants, ranging from students to experts, offered varied responses.

Our three chosen plants influenced how participants engaged with them. We observed everything from gentle petting to energetic drumming on the plants. Interestingly, we found that when the plants were higher up, participants tended to focus more on the trunk.

By grouping interactions based on factors like intensity and duration, we gained a clearer picture of how people approached these musical plants. It turns out that certain interactions, like grasping and pinching, were more common, while others, like sliding, had their own distinct appeal.

In the end, our study not only sheds light on the potential for collaboration between humans and plants but also suggests exciting possibilities for designing interactive plant systems. As we explore this blend of nature and technology, our findings contribute to the ongoing conversation about redefining our relationship with the natural world through the universal language of music.