\section{Bird Species Classification by Sound}

\subsection{Introduction and Objectives}
The classification of bird species based on their vocalizations represents a significant challenge in bioacoustics and sound-based AI. This project aims to develop a sound classification system that can identify bird species from audio recordings, using a setup that captures real-world data from natural environments. As an entry point into sound-based AI, this project provided essential insights into the complexities of audio classification, including the need for robust models capable of handling varied environmental sounds.

The primary objectives of this project were twofold. First, it sought to build an effective classification model trained on a large dataset, specifically the BirdCLEF 2020 dataset, which contains recordings of numerous bird species. This dataset provided a rich source of training data for model development, ensuring the system could generalize across different vocalization patterns. Second, the project aimed to create a simple, user-friendly interface to visualize the classification results. This introduced a user experience (UX) component, emphasizing the importance of making complex AI models accessible to users.